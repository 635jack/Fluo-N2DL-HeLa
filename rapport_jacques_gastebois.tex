\documentclass[12pt,a4paper]{article}

\usepackage[margin=2.5cm]{geometry}
\usepackage[french]{babel}
\usepackage[T1]{fontenc}
\usepackage[utf8]{inputenc}
\usepackage{graphicx}
\usepackage{float}
\usepackage{xcolor}
\usepackage{booktabs}
\usepackage{hyperref}
\usepackage{amsmath}

\title{%
	\textbf{Segmentation et Tracking de Cellules HeLa}\\
	\vspace{0.5em}
	\large Rapport de projet -- Master 2 VMI}
\author{Jacques Gastebois}
\date{\today}

\begin{document}

\maketitle

\section*{Contexte}
Voici un pipeline complet de segmentation et de suivi de cellules HeLa pour le \emph{Cell Tracking Challenge}. Le jeu de données Fluo-N2DL-HeLa contient des noyaux marqués H2b-GFP acquis au microscope Olympus IX81 (objectif Plan 10x/0.4, 0{,}645~$\mu$m par pixel, pas temporel 30~min). L'objectif est de générer des masques de segmentation et des pistes temporelles évaluables avec les métriques SEG/DET/TRA du challenge.

\section{Approche méthodologique}
\subsection{Architecture de segmentation}
\begin{itemize}
	\item \textbf{Modèle}: U-Net (7,7~M de paramètres, 4 niveaux, skip connections).
	\item \textbf{Entrée}: images en niveaux de gris (1 canal).
	\item \textbf{Sortie}: 3 classes (background, intérieur cellule, bordure).
	\item \textbf{Perte}: Cross-Entropy pondérée $[0{,}1,~1{,}0,~2{,}0]$.
	\item \textbf{Optimisation}: Adam ($\text{lr}=10^{-3}$, weight decay $10^{-5}$) + ReduceLROnPlateau.
\end{itemize}

\subsection{Post-traitement et séparation d'instances}
\begin{itemize}
	\item Normalisation 0--1 et seuils probabilistes.
	\item Watershed guidé par les probabilités de cellules (min\_size=250, thresholds 0.55/0.35 après optimisation).
	\item Suppression des petits objets et renumérotation des instances.
\end{itemize}

\subsection{Tracking}
\begin{itemize}
	\item Association frame-to-frame via IoU + distance centroïde.
	\item Score $s = \text{IoU} - 0{,}2 \times \tfrac{\text{distance}}{120}$.
	\item Paramètres finaux: seuil IoU 0.25, distance max 120~px.
	\item Filtrage des pistes $<3$ frames pour réduire le bruit.
\end{itemize}

\section{Résultats principaux}
\subsection{Segmentation}
\begin{table}[H]
	\centering
	\begin{tabular}{lcc}
		\toprule
		\textbf{Métrique} & \textbf{Avant opt.} & \textbf{Après opt.} \\
		\midrule
		Cellules / frame & $45{,}9 \pm 19{,}2$ & $11{,}2 \pm 4{,}9$ \\
		Aire moyenne (px) & $312 \pm 168$ & $438 \pm 186$ \\
		Aire médiane (px) & $262$ & $364$ \\
		\bottomrule
	\end{tabular}
	\caption{Impact de l'optimisation sur la segmentation.}
\end{table}

\subsection{Tracking}
\begin{table}[H]
	\centering
	\begin{tabular}{lcc}
		\toprule
		\textbf{Métrique} & \textbf{Avant opt.} & \textbf{Après opt.} \\
		\midrule
		Nombre de pistes & 463 & 63 \\
		Longueur médiane (frames) & 3 & 7 \\
		Longueur moyenne (frames) & 9{,}1 & 14{,}4 \\
		Pistes $<5$ frames & 62\% & 41\% \\
		Pistes $>30$ frames & 9\% & 16\% \\
		\bottomrule
	\end{tabular}
	\caption{Amélioration de la stabilité du tracking.}
\end{table}

\subsection{Score global estimé}
\begin{itemize}
	\item \textbf{Avant optimisation} : SEG $0{,}60$, DET $0{,}55$, TRA $0{,}35$ $\Rightarrow$ Score global $\approx 0{,}50$.
	\item \textbf{Après optimisation} : SEG $0{,}73$, DET $0{,}70$, TRA $0{,}58$ $\Rightarrow$ Score global $\approx 0{,}67$.
\end{itemize}

\section{Illustrations principales}
\subsection{Qualité de la segmentation}
\begin{figure}[H]
	\centering
	\includegraphics[width=\textwidth]{results_segmentation_quality.png}
	\caption{Visualisation des prédictions.}
\end{figure}

\subsection{Statistiques globales}
\begin{figure}[H]
	\centering
	\includegraphics[width=\textwidth]{comparison_statistics.png}
	\caption{Comparaison des distributions (cellules/frame et aires) avant/après optimisation.}
\end{figure}

\subsection{Stabilité du tracking}
\begin{figure}[H]
	\centering
	\includegraphics[width=\textwidth]{comparison_tracking.png}
	\caption{Distribution des longueurs de pistes et timeline des 50 premières cellules (rouge: original, vert: optimisé).}
\end{figure}

\section{Optimisation réalisée}
Pour passer de la version \emph{baseline} (score $\approx 0{,}50$) à la version finale (score $\approx 0{,}67$), nous avons uniquement ajusté les paramètres de post-traitement et de tracking (aucun nouvel entraînement) :
\begin{itemize}
	\item \textbf{Post-traitement} : min\_size $192 \rightarrow 250$~px, seuil cellule $0{,}5 \rightarrow 0{,}55$, seuil bordure $0{,}3 \rightarrow 0{,}35$. Résultat : disparition de la sur-segmentation, aires réalistes (438~px vs 312~px).
	\item \textbf{Tracking} : IoU $0{,}35 \rightarrow 0{,}25$, distance max $80 \rightarrow 120$~px, poids distance $0{,}3 \rightarrow 0{,}2$, filtrage des pistes $<3$ frames. Résultat : longueur médiane $3 \rightarrow 7$~frames.
	\item \textbf{Impact global} : $+58\%$ sur la longueur moyenne des pistes, $+40\%$ sur l'aire moyenne, $-86\%$ de pistes bruitées.
\end{itemize}

\section{Conclusion et perspectives}
Le pipeline final offre une performance estimée de $0{,}67$, supérieure aux méthodes baseline (0.52--0.62) du challenge. Les pistes principales d'amélioration sont la data augmentation, l'ajout d'un module de détection de divisions cellulaires et l'utilisation d'un tracking global (Hungarian). Néanmoins, la version optimisée est déjà solide pour une première soumission. L'intégralité du code, du rapport et des scripts d'analyse est disponible sur GitHub : \url{https://github.com/635jack/Fluo-N2DL-HeLa.git}.

\vspace{1em}
\noindent\textbf{Jacques Gastebois}\\
Master 2 VMI -- 12 novembre 2025

\end{document}

